\documentclass[12pt,a4paper]{article}
\usepackage[utf8]{inputenc}
\usepackage[german]{babel}
\usepackage[T1]{fontenc}
\usepackage{amsmath}
\usepackage{amsfonts}
\usepackage{amssymb}
\usepackage{minted}

\setcounter{section}{1}
\author{Tronje Krabbe, Julian Deinert}
\title{Labreport 01}
\begin{document}
\maketitle
\tableofcontents
\newpage


\section*{Aufgabe 1}
\addcontentsline{toc}{section}{Aufgabe 1}
\subsection{Hilfe zu Befehlen}
\subsubsection{man ls}
Wir haben uns mit \texttt{man ls} über den \texttt{ls} Befehl informiert.
Interessant sind hierbei beispielsweise die Optionen \texttt{-a, ---all} um auch versteckte Dateien und Ordner anzuzeigen, oder auch \texttt{-lh} um die Größe von Dateien in lesbarer Form anzuzeigen.
\subsubsection{ls \--\--help}
Die Option \texttt{---help} gibt die Kurzreferenz für \texttt{ls} wieder. Diese ist unter Umständen etwas kürzer als die \textit{man page}.
\subsubsection{script}
Der Befehl \texttt{script} schreibt die Ausgaben des Terminals in eine Datei, die standardmäßig mit dem Namen \texttt{typescript} angelegt wird. Dies ist nützlich, da man nicht alle Eingaben mitschreiben muss und später auf diese zurückgreifen kann.
\subsection{Benutzerkonten und -verwaltung}
\begin{enumerate}
    \item Wir benutzen das Programm \texttt{adduser} um den User \texttt{labmate} hinzuzufügen.
    \item Das Programm \texttt{groups} zeigt uns die groups eines users an:
        \begin{minted}{bash}
        $ groups labmate
        labmate : labmate
        \end{minted}
    \item Wir führen aus: \texttt{groupadd labortests}.
    \item \texttt{gpasswd -a labmate labortests}
    \item Wir fügen den User einfach in die Admin Gruppe hinzu: \texttt{gpasswd -a labmate admin}
\end{enumerate}

\subsection{Datei- und Rechteverwaltung}
\begin{enumerate}
    \item \texttt{su labmate}
    \item \texttt{cd; pwd}
    \item \texttt{mkdir labreports}
    \item pico ist ein Alias auf nano. Wir führen also aus:
        \begin{minted}{bash}
        $ touch labreports/bericht1.txt
        $ pico labreports/bericht1.text
        \end{minted}
        Wobei der \texttt{touch} Befehl gar nicht nötig war; nano hätte die Datei beim Speichern einfach erstellt, hätte es sie nicht gegeben.
    \item \texttt{chmod 0660 bericht1.txt}
    \item \texttt{wget -O labreports/index.html http://www.uni-hamburg.de/index.html}
    \item
        \begin{minted}{bash}
        $ cd
        $ chmod 0660 labreports
        \end{minted}
        Hiernach haben wir keine execute-Rechte für das directory, können also
        nicht mit ihm interagieren. Die Lösung ist:
        \texttt{chmod 0770 labreports}
    \item Wir haben nicht die nötigen Rechte, um in roots home-Verzeichnis zu wechseln.
    \item
        \begin{minted}{bash}
        # sudo, da wir admin Rechte für das Schreiben in /opt benötigen
        $ sudo mkdir /opt/test
        $ cd /opt
        $ sudo chown labmate:admin test
        $ chmod 0770 test
        \end{minted}
        Der \texttt{chmod}-Befehl bewirkt, dass \texttt{labmate} und alle user in
        der Gruppe der Datei jetzt read-, write- und execute-Permissions haben.
    \item \texttt{cp ~/labreports/index.html /opt/test/}
    \item Wir benutzen die Programme \texttt{chown} und \texttt{chmod}:
        \begin{minted}{bash}
        $ chown labmate:user index.html
        $ chmod 0640 index.html
        \end{minted}
    \item \texttt{exit} oder einfach Ctrl+D
    \item Das Auslesen der Datei gelingt, da wir ja read-Permissions haben.
    \item Das Öffnen in nano funktioniert, das Speichern nach dem Editieren aber nicht,
        da wir ja keine write-Permissions haben.
    \item \texttt{cp /opt/test/index.html /opt/test/userindex.html}
    \item Das Editieren der Datei gelingt, dass sie jetzt uns, also dem user \texttt{user}
        gehört.
    \item Ja, wir können die Datei \textit{index.html} löschen; beim Löschen kommt
        es auf die write- und execute-Rechte des parent-directory an, nicht auf die Rechte
        zu der Datei selbst.
\end{enumerate}
\subsection{Administration und Aktualisierung}
\texttt{apt-get update} aktualisiert die lokale Paket-Datenbank, und die Option
\texttt{upgrade} installiert die neusten Versionen der bereits installierten Pakete.
\\
\begin{verbatim}
_________ 
< NetSec! >
 --------- 
        \   ^__^
         \  (oo)\_______
            (__)\       )\/\
                ||----w |
                ||     ||
\end{verbatim}
\subsection{Prozesse und Prozessverwaltung}
\begin{enumerate}
    \item \texttt{ps} gibt diverse Informationen zu laufenden Prozessen aus. \texttt{top}
        gibt eine grafische Übersicht über laufende Prozesse, sowie Nutzung des
        Arbeitsspeichers.
    \item \texttt{\$ top} % kek
    \item Nach Ausführung des Befehls \texttt{cat /dev/urandom} zeigt \texttt{top} %kek
        für den Prozess \texttt{cat}, der von \textit{labmate} ausgeführt wurde, eine
        erhöhten CPU-Last von 8\% sowie 0.1\% RAM-Nutzung an, was für \texttt{cat} sehr
        viel ist.
    \item Wir haben nicht die nötigen Permissions, um den Prozess eines anderen Users zu
        killen. Als \textit{root} können wir theoretisch alles killen (mit Ausnahme
        von \textit{init}, also auch den \texttt{cat}-Prozess von \textit{labmate}.
    \item Nicht direkt; \textit{labmate} darf nicht \texttt{reboot} aufrufen. Da
        \textit{labmate} aber in der \textit{admin}-Gruppe ist, kann er
        \texttt{sudo reboot} ausführen.
    \item
        \begin{minted}{bash}
        $ crontab -e
        # m h dom mon dow  command
        */5 * * * * /bin/date >> /home/labmate/zeitstempel.txt
        \end{minted}
\end{enumerate}
\subsection{VMware-Tools installieren}
\begin{itemize}
    \item Der \texttt{tar}-Befehl bekommt folgende Optionen übergeben:
        \texttt{xzf <Dateiname>}, optional noch \texttt{v}. Diese bewirken
        (in der Reihenfolge in der sie auftreten): extrahieren des Tarballs,
        filtern durch \texttt{gzip}, benutzen der angegebenen Datei
        (anstelle von \texttt{stdin}). Und \texttt{v} erhöht die Ausführlichkeit
        der nach \texttt{stdout} geschriebenen Meldungen.
    \item Die Fehlermeldung gab zum Ausdruck, dass wir nicht die nötigen Berechtigungen
        haben. Dies ist mit \texttt{sudo} behebbar.
\end{itemize}
\subsection{Bedienung von VMware}
\end{document}

