\documentclass[12pt,a4paper]{article}
\usepackage[utf8]{inputenc}
\usepackage[german]{babel}
\usepackage[T1]{fontenc}
\usepackage{amsmath}
\usepackage{amsfonts}
\usepackage{amssymb}
\usepackage{minted}

\setcounter{section}{1}
\author{Tronje Krabbe, Julian Deinert}
\title{Labreport 01}
\begin{document}
\maketitle
\tableofcontents
\newpage


\section*{Aufgabe 1}
\subsection{Hilfe zu Befehlen}
\addcontentsline{toc}{section}{Aufgabe 1}
\subsubsection{man ls}
Wir haben uns mit \texttt{man ls} über den \texttt{ls} Befehl informiert.
Interessant sind hierbei beispielsweise die Optionen \texttt{-a, ---all} um auch versteckte Dateien und Ordner anzuzeigen, oder auch \texttt{-lh} um die Größe von Dateien in lesbarer Form anzuzeigen.
\subsubsection{ls \--\--help}
Die Option \texttt{---help} gibt die Kurzreferenz für \texttt{ls} wieder. Diese ist unter Umständen etwas kürzer als die \textit{man page}.
\subsubsection{script}
Der Befehl \texttt{script} schreibt die Ausgaben des Terminals in eine Datei, die standardmäßig mit dem name \texttt{typescript} angelegt wird. Dies ist nützlich, da man nicht alle Eingaben mitschreiben muss und später auf diese zurückgreifen kann.
\subsection{Benutzerkonten und -verwaltung}
\subsection{Datei- und Rechteverwaltung}
\subsection{Administration und Aktualisierung}
\subsection{Prozesse und Prozessverwaltung}
\subsection{VMware-Tools installieren}
\subsection{Bedienung von VMware}





\end{document}

