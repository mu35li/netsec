\documentclass[12pt,a4paper]{article}
\usepackage[utf8]{inputenc}
\usepackage[german]{babel}
\usepackage[T1]{fontenc}
\usepackage{amsmath}
\usepackage{amsfonts}
\usepackage{amssymb}
\usepackage{minted}

\setcounter{section}{1}
\author{Tronje Krabbe, Julian Deinert}
\title{Labreport 02}
\begin{document}
\maketitle
\tableofcontents
\newpage


\section*{Aufgabe 1}
\addcontentsline{toc}{section}{Aufgabe 1}
\subsection{Zugriff auf /etc/passwd und /etc/shadow des Webservers}
\begin{itemize}
\item Wir haben den Ordner mit dem \textbf{File Browser} in unseren vmware-Ordner kopiert.
\item Wir änderen die Einstellungen unserer VM so, dass das grml-Abbild im virtuellen CD/DVD-Laufwerk liegt und setzen den Haken bei \textit{connect on power on}. Letzteres sorgt dafür, dass das image geladen wird, bevor Ubuntu gestartet wird.
\item Nach dem Starten von \textit{grml} mounten wir die \textit{Root-Partition} nach \texttt{/mnt}. Mit \texttt{lsblk} ergibt sich aus dem output
\begin{verbatim}
NAME   MAJ:MIN RM   SIZE RO TYPE MOUNTPOINT
fd0      2:0    1     4K  0 disk
sda      8:0    0    20G  0 disk 
|-sda1   8:1    0  19.1G  0 part 
|-sda2   8:2    0     1K  0 part 
|-sda5   8:3    0   895M  0 part

\end{verbatim}
dass die \textit{Root-Partition} auf \texttt{sda1} liegt.
Wir mounten diese also mit \texttt{sudo mount /dev/sda1 /mnt} in den Ordner \texttt{/mnt}.

\item Die Datei \textit{passwd} enthält für jeden User eine Zeile, die aus folgenden Feldern besteht:
\begin{itemize}
\item Dem Login-Namen
\item Dem verschlüsseltem Password des Users (optional)
\item Der User-ID
\item Der Gruppen-ID
\item Dem User-Namen oder einem Kommentar
\item Dem \textit{home directory} des Users
\item Der \textit{shell} des Users (optional)

\end{itemize}

\item Die Datei \textit{shadow} speichert die gehashten Passwörter der User im folgenden Format:\\
<login name>:<encrypted password>:\allowbreak<date of last password change>:\allowbreak<minimum password age>:\allowbreak<maximum password age>:\allowbreak<password warning period>:\allowbreak<password inactivity period>:\allowbreak<account expiration date>:\allowbreak<reserved field>

In beiden Datien gibt es neben diversen Usern für Services und Programme die User \texttt{root}, \texttt{georg} und \texttt{webadmin}.

\item Der User \texttt{georg} gehört der Gruppe \texttt{admin} an. \texttt{webadmin} gehört den Gruppen \texttt{adm, dialout, cdrom, plugdev, lpadmin} \\  sowie \texttt{sambashare} an.
\end{itemize}

\subsection{Auslesen von Kennwörtern}
\begin{itemize}
\item Ein gesaltetes, gehashtes Passwort ist ein Passwort, das erst mit einem sogenannten Salt kombiniert und danach mit einer Hash-Funktion gehasht wurde. Eine Hash-Funktion verändert ihren Input deterministisch so, dass das Ergebnis nicht (ohne Weiteres) zurück zum Input transformiert werden kann. Der Salt kann eine beliebige Zeichenkombination sein, dessen Nutzung die Anwendung von Rainbow-Tables verhindern soll. Insbesondere erschwert der Hash das Reversieren des gehashten Wertes zum Original.
\item Die \texttt{---incremental} Option führt einen Brute-Force-Angriff auf das Passwort auf, was sehr, sehr langsam ist aufgrund der vielen möglichen Kombinationen von Zeichen.
\item Der Wörterbuch-Angriff funktioniert sehr viel schneller, und das Passwort lautet \textit{mockingbird}.
\end{itemize}

\subsection{Setzen von neuen Kennwörtern}
Das Passwort von \textit{georg} konnte nicht mit \texttt{john} ermittelt werden, da es nicht in unserem (und daher wahrscheinlich auch in keinem anderen) Wörterbuch vorkommt.

\setcounter{section}{2}
\setcounter{subsection}{0}
\section*{Aufgabe 2}
\addcontentsline{toc}{section}{Aufgabe 2}
\subsection{Angriffe mit Hashdatenbanken und Rainbow-Tables}
Die ermittelten Kennwörter sind:
\begin{itemize}
\item ente
\item ball
\item borkeni
\item ulardi
\item avanti
\end{itemize}
Zur Idee von Black Hat:
es gibt 62 alphanumerische Zeichen (26 Klein- und 26 Großbuchstaben, sowie 10 Ziffern).
Das ergibt $$ \sum_1^{n=7} 62^i = 3.579.345.993.194 $$ mögliche Hashes, jeder mit einer
Größe von 32 Byte, bzw. 33 Byte inklusive des Separators, was insgesamt $118,1$ TB belegt.
Wer hier pedantisch sein möchte, könnte bemerken, dass wir ein Byte zu viel eingerechnet
haben, da nach dem letzten Eintrag kein Separator benötigt wird. Trotzdem ist dies ein
exorbitanter Speicherplatzbedarf, der uns auch nur weiterhilft, wenn das Passwort
tatsächlich nur aus alphanumerischen Zeichen besteht. Die Rainbow-Tables dagegen belegen
lediglich $12,2$ GB.

\subsection{Eigener Passwort-Cracker}
Wir haben unseren Passwort-Cracker in Python geschrieben; er benutzt \texttt{hashlib},
für die \textit{md5}-Funktion, und \texttt{itertools}, um alle Permutationen
des Alphabets bis zur Länge 6 zu finden.
Das Ergebnis ist: s1v3s.
Der Source-Code befindet sich im Appendix.


\end{document}

