\documentclass[10pt,a4paper]{article}
\usepackage[utf8]{inputenc}
\usepackage[german]{babel}
\usepackage[T1]{fontenc}
\usepackage{amsmath}
\usepackage{amsfonts}
\usepackage{amssymb}
\usepackage{minted}
\usepackage[left=1.2cm, right=1.2cm, top=1.5cm, bottom=2cm]{geometry}
\usepackage{url}
\usepackage{graphicx}


\setcounter{section}{1}
\author{Julian Deinert, Tronje Krabbe}
\title{Labreport 06}
\linespread{1.0}
\begin{document}
\maketitle
\tableofcontents
\thispagestyle{empty}
\newpage
\setcounter{page}{1}


\section*{1. TODO}
\addcontentsline{toc}{section}{1. TODO}
\setcounter{subsection}{1}

\subsection{}
TODO

\setcounter{section}{2}
\section*{2. TODO}
\addcontentsline{toc}{section}{2. TODO}
\setcounter{subsection}{0}
\subsection{}
TODO

\setcounter{section}{3}
\section*{3. Unsichere selbstenwtickelte Verschlüsselungsalgorithmen}
\addcontentsline{toc}{section}{3. Unsichere selbstenwtickelte Verschlüsselungsalgorithmen}
\setcounter{subsection}{0}
\subsection{BaziCrypt}
Die letzten 10 Bytes des Ciphertexts sind exakt der Key. Oder, genauergesagt, der Key XOR Null,
was den Key unverändert lässt. Jetzt ist es sehr einfach, die Dateien zu entschlüsseln. Siehe
unser Skript im Appendix. Die drei Plaintexte sind:
\begin{quote}[n01.txt.enc]
Hallo Peter. Endlich koennen wir geheim kommunizieren! Bis bald, Max
\end{quote}
\begin{quote}[n02.txt.enc]
Hi Max! Super, Sicherheitsbewusstsein ist ja extrem wichtig! Schoene Gruesse, Peter.
\end{quote}
\begin{quote}[n03.txt.enc]
Hi Peter, hast du einen Geheimtipp fuer ein gutes Buch fuer mich? Gruss, Max
\end{quote}

\subsection{AdvaziCrypt - Denksport}
Da \textit{PKCS7}-Padding blockweise passiert, und der Key 10 Bytes lang ist, können
wir davon ausgehen, dass die letzten 10 Bytes des Ciphertexts sich so ergeben:
\begin{align*}
    \text{key} \oplus 0x10 = \text{ciphertext}
\end{align*}
Um nun also den Key zu ermitteln, rechnen wir einfach:
\begin{align*}
    \text{ciphertext} \oplus 0x10 = \text{key}
\end{align*}

\newpage
\section*{Appendix}
\subsection{3. Skripte}
\inputminted{python}{crack_bazi.py}

\end{document}
