\documentclass[10pt,a4paper]{article}
\usepackage[utf8]{inputenc}
\usepackage[german]{babel}
\usepackage[T1]{fontenc}
\usepackage{amsmath}
\usepackage{amsfonts}
\usepackage{amssymb}
\usepackage{minted}
\usepackage[left=2cm, right=2cm]{geometry}
\usepackage{url}


\setcounter{section}{1}
\author{Tronje Krabbe, Julian Deinert}
\title{Labreport 02}
\begin{document}
\maketitle
\tableofcontents
\newpage


\section*{Aufgabe 1 HTTP}
\addcontentsline{toc}{section}{Aufgabe 1 HTTP}
\subsection{Telnet}
Wir haben versucht uns mit dem Befehl \texttt{telnet} mit dem angegebenen Server zu verbinden.
\begin{minted}{bash}
$ telnet www.inf.uni-hamburg.de 80                                                                       
Trying 134.100.56.130...
Connected to www.inf.uni-hamburg.de.
Escape character is '^]'.

GET /de/inst/ab/svs/home.html HTTP/1.1
\end{minted}
Als Antwort auf unseren GET-Request erhalten wir eine Website mit Returncode \texttt{302 Found}, die uns sagt, dass das Dokument nur mittels \textit{https} erreichbar ist. Da Telnet kein https kann, greifen wir auf openssl zurück.
\begin{minted}{bash}
$ openssl s_client -connect www.inf.uni-hamburg.de:443

GET /de/inst/ab/svs/home.html HTTP/1.1

\end{minted}
Wir erhalten den HTTP-Fehlercode \texttt{400 Bad Request} zurück. Dementsprechend können wir auch keine CSS-Dateien anfordern.

\setcounter{section}{2}
\setcounter{subsection}{0}
\section*{Aufgabe 2 SMTP}
\addcontentsline{toc}{section}{Aufgabe 2 SMTP}
\subsection{Mail Spoofing}
Wir verbinden uns mittels \textit{Netcat} mit dem Mailserver \texttt{mailhost.informatik.uni-hamburg.de} auf Port 25. Nach dem wir die Felder \texttt{FROM}, \texttt{RCPT TO} sowie \texttt{DATA} gesetzt haben wird unsere Mail erfolgreich versendet.
Der Empfänger kann anhand des Quelltextes erkennen, dass die Mail nicht von einem \textit{Authenticated sender} geschickt wurde.\\
Es gibt keinen Unterschied ziwschen einer gespooften gmail oder informatik.uni-hamnburg Adresse, da beide Mails direkt beim Mailhost eingereicht wurden.

\setcounter{section}{3}
\setcounter{subsection}{0}
\section*{Aufgabe 3 License Server}
\addcontentsline{toc}{section}{Aufgabe 2 License Server}
\subsection{DNS Spoofing}
Wir haben uns mit \textit{netcat} zum Server verbunden und werden aufgefordert einen von 4 validen Commands einzugeben. Diese sind:
\begin{itemize}
\item help
\item serial
\item version
\item quit
\end{itemize}
Wenn wir eine Serial angeben, die zufällig keine gültige ist, bekommen wir die Meldung \texttt{SERIAL\_VALID=0} zurück. Anhand dieser Information erstellen wir unseren TCP server so, dass bei jeder Serial die Meldung \texttt{SERIAL\_VALID=1} zurückgegeben wird.
\subsection{Eigener License Server}
\begin{minted}{python}
#!/usr/bin/env python
import socketserver as ss

class LeetTCPHandler(ss.BaseRequestHandler):
    def handle(self):
        self.data = self.request.recv(1024).strip()
        print(self.data)
        if self.data == b'VERSION':
            self.request.sendall(b"Numeric Serial Server Validation System 2.1a")
        elif self.data[:6] == b'SERIAL':
            self.request.sendall(b"SERIAL_VALID=1")
        else:
            self.request.sendall(b"Invalid command")


if __name__ == "__main__":
    HOST, PORT = "localhost", 1337

    server = ss.TCPServer((HOST, PORT), LeetTCPHandler)

    server.serve_forever()
\end{minted}

Außerdem ergänzen wir unser Host-File um diese Zeile:\\
\texttt{127.0.0.1 license-server.svslab localhost} \\
Nun können wir das Java-Programm ausführen, und es verbindet sich mit unserem
eigenen TCP-Server, und akzeptiert eine beliebige Zahl als License-Key:
\begin{verbatim}
Your license has expired - please enter new license key!
To order your new key for just 999.99$ call +1-555-we-rule

Key: 123
Numeric Serial Server Validation Sy
Thanks for purchasing a new license!
\end{verbatim}

\subsection{Verhinderung des Angriffs}
Eine naive Möglichkeit, den Spoof zu entdecken, wäre einfach, ein weniger offensichtiliches
Schlüsselwort als Ausgabe nach erfolgreicher Überprüfung zu wählen. Momentan gibt
der Server \textit{SERIAL\_VALID=0} zurück, wenn der Key falsch ist.
Daraus ist leicht zu folgern, dass \textit{SERIAL\_VALID=1} die Ausgabe ist, die
bei einem richtigen Key gegeben wird. Dies könnte man ändern um den Angriff
zu erschweren.

Eine sicherere Methode wäre, Public-Key-Cryptographie zu nutzen. Das Programm wird
mit dem public-key des Servers sowie seinem eigenen private-key kompiliert.
Der Server kennt den public-key des Programms. Die Kommunikation geschieht so,
dass alles, was das Programm an den Server sendet, mit seinem public-key verschlüsselt ist.
Der Server entschlüsselt die erhaltenen Daten mit seinem private-key, und
verschlüsselt seine Ausgabe mit dem public-key des Programms.

Auch hier liegt leider eine Schwierigkeit; hat jede Kopie des Programms das gleiche
Key-Paar? Wenn nicht, woher weiß der Server, welchen Key er benutzen muss?
Außerdem könnte man den Speicher des Programms untersuchen, während es läuft,
oder einen Decompiler benutzen, um die Keys zu finden.

\section{Aufgabe 4 License-Server}
\setcounter{subsection}{0}
\subsection{}
Unser Programm ist im Anhang zu finden.

\subsection{}
Es gibt mehrere Möglichkeiten, so einen Service gegen Brute-Force Angriffe zu schützen.
Von Anfang an ist es eine gute Idee, Authentifizierungsversuche zeitlich zu begrenzen.
Wenn nur eine Validierung pro Sekunde zugelassen wird, merkt ein User mit einem validen
Key praktisch keinen Unterschied, ein Brute-Force Angriff hingegen dauert dann deutlich länger.
Weiterhin kann man IPs, die eine gewissen Anzahl Fehlversuche hinter sich haben, weiter
verlangsamt bedienen, oder gar ganz blockieren.

\subsection{}
Wir haben folgende Zahlen haben wir rausgefunden:
90877300,
31337000,
21935900,
62674000,
18802200.
Wenn man den größten gemeinsamen Teiler dieser Zahlen berechnet,
erhält man 3133700. Und tatsächlich akzeptiert der Server alle Zahlen,
die nach der Formel $(n \cdot 3133700) \mod 10^8$ generiert wurden.

\section{Aufgabe 5 Implementieren eines TCP-Chats}
\setcounter{subsection}{0}
\subsection{}
Die vom Server ausgestrahlten URLs sind
\url{http://www.oracle.com/technetwork/java/socket-140484.html} und
\url{https://code.google.com/archive/p/example-of-servlet/source/default/source}.

\subsection{}
Wir kompilieren die Quelltext-Dateien \texttt{ClientWorker.java},
\texttt{SocketThrdServer.java} und \texttt{SocketClient.java},
um den Server und die Client ausführen zu können.
Wenn wir in den Quelltext schauen, finden wir raus, dass der Server
auf Port 4444 servt. Wenn wir ihn mit \texttt{telnet} ansprechen,
wird unser Input immer noch einmal zurückgeschickt, und im GUI des Servers
angezeigt. Da wir die gethreadete Variante nehmen, können wir auch gleichzeitig
mit dem \texttt{SocketClient} Programm `chatten'.

\subsection{}
Zuerst sammeln wir alle Worker, die wir erstellen, in einer
\texttt{ArrayList}:
\begin{minted}{java}
private List<ClientWorker> workers = new ArrayList<ClientWorker>();
\end{minted}
Dann wird jedes Mal, wenn ein Thread gestartet wird, das Worker Objekt
in diese Liste hinzugefügt.
Wenn ein Worker Objekt erzeugt wird, bekommt es auch die Instanz
des Servers übergeben (\texttt{this}). Sendet der Client jetzt eine Nachricht,
kann der Worker am Server \textit{broadcast} aufrufen:
\begin{minted}{java}
server.broadCast(this, line);
\end{minted}
\newpage
% this line would be the last on the page without the 'newpage':
Und broadcast sieht so aus:
\begin{minted}{java}
public void broadCast(ClientWorker client, String message) {
    if (message != null) {
        for (ClientWorker worker : workers) {
            if (!(worker.equals(client))) {
                System.out.println("sending Message "+message+" to "+client.toString());
                worker.sendMessage(message);
            }
        }
    }
}
\end{minted}

\subsection{}
Im Grunde erweitern wir die ClientWorker Klasse um diese beiden Methoden:
\begin{minted}{java}
private boolean register(String username, String password){
    Useradmin useradminObj = new Useradmin();
    useradminObj.addUser(username, password);
    return true;
}

private boolean login(String username, String password){
    Useradmin useradminObj = new Useradmin();
    boolean check = useradminObj.checkUser(username, password);
    if (check == true) {
        this.isLoggedIn = true;
        return true;
    }else{
        return false;
    }
}
\end{minted}
Diese werden in der \texttt{run()} Methode aufgerufen,
wenn der User das Kommando ``register'' schickt. Der komplette
Code findet sich im Appendix.

\subsection{}
TODO


\newpage
\section{Appendix}
\subsection{SocketThrd.java}
\begin{minted}[linenos, fontsize=\small]{java}
import java.awt.BorderLayout;
import java.awt.Color;
import java.awt.event.WindowAdapter;
import java.awt.event.WindowEvent;
import java.awt.event.WindowListener;
import java.io.IOException;
import java.net.ServerSocket;
import java.util.*;

import javax.swing.JFrame;
import javax.swing.JLabel;
import javax.swing.JPanel;
import javax.swing.JTextArea;

class SocketThrdServer extends JFrame{

    JLabel label = new JLabel("Text received over socket:");
    JPanel panel;
    JTextArea textArea = new JTextArea();
    ServerSocket server = null;
    private List<ClientWorker> workers = new ArrayList<ClientWorker>();

    SocketThrdServer(){ //Begin Constructor
        panel = new JPanel();
        panel.setLayout(new BorderLayout());
        panel.setBackground(Color.white);
        getContentPane().add(panel);
        panel.add("North", label);
        panel.add("Center", textArea);
    } //End Constructor

    public void listenSocket(){
        try{
            server = new ServerSocket(4444); 
        } catch (IOException e) {
            System.out.println("Could not listen on port 4444");
            System.exit(-1);
        }
        while(true){
            ClientWorker w;
            try{
                w = new ClientWorker(server.accept(), textArea, this);
                System.out.println(w.toString());
                Thread t = new Thread(w);
                t.start();
                workers.add(w);
                System.out.println("Added worker to list");
                for (ClientWorker worker : workers) {
                    System.out.println("worker:"+worker.toString());
                }
            } catch (IOException e) {
                System.out.println("Accept failed: 4444");
                System.exit(-1);
            }
        }
    }

    protected void finalize(){
        //Objects created in run method are finalized when 
        //program terminates and thread exits
        try{
            server.close();
        } catch (IOException e) {
            System.out.println("Could not close socket");
            System.exit(-1);
        }
    }

    public static void main(String[] args){
        SocketThrdServer frame = new SocketThrdServer();
        frame.setTitle("Server Program");
        WindowListener l = new WindowAdapter() {
            public void windowClosing(WindowEvent e) {
                System.exit(0);
            }
        };
        frame.addWindowListener(l);
        frame.pack();
        frame.setVisible(true);
        frame.listenSocket();
    }

    public void broadCast(ClientWorker client, String message) {
        if (message != null) {
            for (ClientWorker worker : workers) {
                if (!(worker.equals(client))) {
                    System.out.println("sending Message "+message+" to "+client.toString());
                    worker.sendMessage(message);
                }
            }
        }
    }

    public void unlistenWorker(ClientWorker client) {

        workers.remove(client);
    }
}
\end{minted}

\newpage
\subsection{ClientWorker.java}
\begin{minted}[linenos, fontsize=\small]{java}
import java.awt.Color;
import java.awt.BorderLayout;
import java.awt.event.*;
import javax.swing.*;

import java.io.*;
import java.net.*;

class ClientWorker implements Runnable {
    private Socket client;
    private JTextArea textArea;
    private SocketThrdServer server;
    private boolean isLoggedIn = false;
    private int failedAttempts = 0;
    ClientWorker(Socket client, JTextArea textArea, SocketThrdServer server) {
        this.client = client;
        this.textArea = textArea;
        this.server = server;
    }
    public void run(){
        String line;
        BufferedReader in = null;
        PrintWriter out = null;
        try{
            in = new BufferedReader(new InputStreamReader(client.getInputStream()));
            out = new PrintWriter(client.getOutputStream(), true);
        } catch (IOException e) {
            System.out.println("in or out failed");
            System.exit(-1);
        }

        while(true){
            try{
                line = in.readLine();
                System.out.println(line);
                if (line.equals("register")) {
                    out.println("Please enter a username");
                    String username = in.readLine();
                    out.println("Please enter a password");
                    String password = in.readLine();
                    textArea.append("User wants to register with username: "+username+" and password "+password+"\n");
                    boolean register = this.register(username, password);
                    if (register == true) {
                        out.println("registered successfully");
                    }
                }else if(line.equals("login")){
                    if (failedAttempts >=3) {
                        out.println("Too many failed attempts");
                    }else{
                        out.println("Please enter a username");
                        String username = in.readLine();
                        out.println("Please enter a password");
                        String password = in.readLine();
                        textArea.append("User wants to login with username: "+username+" and password "+password+"\n");
                        boolean login = this.login(username, password);
                        if (login == true) {
                            out.println("logged in successfully");
                        }else{
                            this.failedAttempts ++;
                            if (failedAttempts >=3) {
                                out.println("Login has been blocked");
                            }
                            out.println("login failed");
                        }
                    }
                }else{
                    //Send data back to client
                    if (line != null && this.isLoggedIn) {
                        out.println(line);
                        textArea.append(line+"\n");
                        server.broadCast(this, line);
                    }else{
                        out.println("Please login to chat");
                    }
                }
            } catch (IOException e) {
                System.out.println("Read failed");
                System.exit(-1);
            }
        }
    }
    public void sendMessage(String message){
        PrintWriter out = null;
        try{
            out = new PrintWriter(client.getOutputStream(), true);
            System.out.println("out created");
            out.println(message);
        } catch (IOException e) {
            System.out.println("out failed");
        }
    }
    protected void finalize(){
        //Objects created in run method are finalized when 
        //program terminates and thread exits
        server.unlistenWorker(this);
    }
    private boolean register(String username, String password){
        Useradmin useradminObj = new Useradmin();
        useradminObj.addUser(username, password);
        return true;
    }
    private boolean login(String username, String password){
        Useradmin useradminObj = new Useradmin();
        boolean check = useradminObj.checkUser(username, password);
        if (check == true) {
            this.isLoggedIn = true;
            return true;
        }else{
            return false;
        }
    }
}
\end{minted}
\end{document}
