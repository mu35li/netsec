\documentclass[10pt,a4paper]{article}
\usepackage[utf8]{inputenc}
\usepackage[german]{babel}
\usepackage[T1]{fontenc}
\usepackage{amsmath}
\usepackage{amsfonts}
\usepackage{amssymb}
\usepackage{minted}
\usepackage[left=2cm, right=2cm]{geometry}
\usepackage{url}


\setcounter{section}{1}
\author{Tronje Krabbe, Julian Deinert}
\title{Labreport 02}
\begin{document}
\maketitle
\tableofcontents
\newpage


\section*{Aufgabe 1}
\addcontentsline{toc}{section}{Aufgabe 1}
\subsection{Telnet}
Wir haben versucht uns mit dem Befehl \texttt{telnet} mit dem angegebenen Server zu verbinden.
\begin{minted}{bash}
$ telnet www.inf.uni-hamburg.de 80                                                                       
Trying 134.100.56.130...
Connected to www.inf.uni-hamburg.de.
Escape character is '^]'.

GET /de/inst/ab/svs/home.html HTTP/1.1
\end{minted}
Als Antwort auf unseren GET-Request erhalten wir eine Website mit Returncode \texttt{302 Found}, die uns sagt, dass das Dokument nur mittels \textit{https} erreichbar ist. Da Telnet kein https kann, greifen wir auf openssl zurück.
\begin{minted}{bash}
$ openssl s_client -connect www.inf.uni-hamburg.de:443

GET /de/inst/ab/svs/home.html HTTP/1.1

\end{minted}
Wir erhalten den HTTP-Fehlercode \texttt{400 Bad Request} zurück. Dementsprechend können wir auch keine CSS-Dateien anfordern.

\end{document}
