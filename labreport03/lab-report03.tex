\documentclass[10pt,a4paper]{article}
\usepackage[utf8]{inputenc}
\usepackage[german]{babel}
\usepackage[T1]{fontenc}
\usepackage{amsmath}
\usepackage{amsfonts}
\usepackage{amssymb}
\usepackage{minted}
\usepackage[left=2cm, right=2cm]{geometry}
\usepackage{url}


\setcounter{section}{1}
\author{Tronje Krabbe, Julian Deinert}
\title{Labreport 02}
\begin{document}
\maketitle
\tableofcontents
\newpage


\section*{Aufgabe 1 HTTP}
\addcontentsline{toc}{section}{Aufgabe 1 HTTP}
\subsection{Telnet}
Wir haben versucht uns mit dem Befehl \texttt{telnet} mit dem angegebenen Server zu verbinden.
\begin{minted}{bash}
$ telnet www.inf.uni-hamburg.de 80                                                                       
Trying 134.100.56.130...
Connected to www.inf.uni-hamburg.de.
Escape character is '^]'.

GET /de/inst/ab/svs/home.html HTTP/1.1
\end{minted}
Als Antwort auf unseren GET-Request erhalten wir eine Website mit Returncode \texttt{302 Found}, die uns sagt, dass das Dokument nur mittels \textit{https} erreichbar ist. Da Telnet kein https kann, greifen wir auf openssl zurück.
\begin{minted}{bash}
$ openssl s_client -connect www.inf.uni-hamburg.de:443

GET /de/inst/ab/svs/home.html HTTP/1.1

\end{minted}
Wir erhalten den HTTP-Fehlercode \texttt{400 Bad Request} zurück. Dementsprechend können wir auch keine CSS-Dateien anfordern.

\setcounter{section}{2}
\setcounter{subsection}{0}
\section*{Aufgabe 2 SMTP}
\addcontentsline{toc}{section}{Aufgabe 2 SMTP}
\subsection{Mail Spoofing}
Wir verbinden uns mittels \textit{Netcat} mit dem Mailserver \texttt{mailhost.informatik.uni-hamburg.de} auf Port 25. Nach dem wir die Felder \texttt{FROM}, \texttt{RCPT TO} sowie \texttt{DATA} gesetzt haben wird unsere Mail erfolgreich versendet.
Der Empfänger kann anhand des Quelltextes erkennen, dass die Mail nicht von einem \textit{Authenticated sender} geschickt wurde.\\
Es gibt keinen Unterschied ziwschen einer gespooften gmail oder informatik.uni-hamnburg Adresse, da beide Mails direkt beim Mailhost eingereicht wurden.

\setcounter{section}{3}
\setcounter{subsection}{0}
\section*{Aufgabe 3 License Server}
\addcontentsline{toc}{section}{Aufgabe 2 License Server}
\subsection{DNS Spoofing}
Wir haben uns mit \textit{netcat} zum Server verbunden und werden aufgefordert einen von 4 validen Commands einzugeben. Diese sind:
\begin{itemize}
\item help
\item serial
\item version
\item quit
\end{itemize}
Wenn wir eine Serial angeben, die zufällig keine gültige ist, bekommen wir die Meldung \texttt{SERIAL\_VALID=0} zurück. Anhand dieser Information erstellen wir unseren TCP server so, dass bei jeder Serial die Meldung \texttt{SERIAL\_VALID=1} zurückgegeben wird.
\subsection{Eigener License Server}

\end{document}
