\documentclass[12pt,a4paper]{article}
\usepackage[utf8]{inputenc}
\usepackage[german]{babel}
\usepackage[T1]{fontenc}
\usepackage{amsmath}
\usepackage{amsfonts}
\usepackage{amssymb}
\usepackage{minted}

\setcounter{section}{1}
\author{Tronje Krabbe, Julian Deinert}
\title{Labreport 02}
\begin{document}
\maketitle
\tableofcontents
\newpage


\section*{Aufgabe 1}
\addcontentsline{toc}{section}{Aufgabe 1}
\subsection{Zugriff auf /etc/passwd und /etc/shadow des Webservers}
\begin{itemize}
\item Wir haben den Ordner mit dem \textbf{File Browser} in unseren vmware-Ordner kopiert.
\item Wir änderen die Einstellungen unserer VM so, dass das grml-Abbild im virtuellen CD/DVD-Laufwerk liegt und setzen den Haken bei \textit{connect on power on}. Letzteres sorgt dafür, dass das image geladen wird, bevor Ubuntu gestartet wird.
\item Nach dem Starten von \textit{grml} mounten wir die \textit{Root-Partition} nach \texttt{/mnt}. Mit \texttt{lsblk} ergibt sich aus dem output
\begin{verbatim}
NAME   MAJ:MIN RM   SIZE RO TYPE MOUNTPOINT
fd0      2:0    1     4K  0 disk
sda      8:0    0    20G  0 disk 
|-sda1   8:1    0  19.1G  0 part 
|-sda2   8:2    0     1K  0 part 
|-sda5   8:3    0   895M  0 part

\end{verbatim}
dass die \textit{Root-Partition} auf \texttt{sda1} liegt.
Wir mounten diese also mit \texttt{sudo mount /dev/sda1 /mnt} in den Ordner \texttt{/mnt}.

\item Die Datei \textit{passwd} enthält für jeden User eine Zeile, die aus folgenden Feldern besteht:
\begin{itemize}
\item Dem Login-Namen
\item Dem verschlüsseltem Password des Users (optional)
\item Der User-ID
\item Der Gruppen-ID
\item Dem User-Namen oder einem Kommentar
\item Dem \textit{home directory} des Users
\item Der \textit{shell} des Users (optional)

\end{itemize}

\item Die Datei \textit{shadow} speichert die gehashten Passwörter der User im folgenden Format:\\
<login name>:<encrypted password>:\allowbreak<date of last password change>:\allowbreak<minimum password age>:\allowbreak<maximum password age>:\allowbreak<password warning period>:\allowbreak<password inactivity period>:\allowbreak<account expiration date>:\allowbreak<reserved field>

\item In beiden Datien gibt es neben diversen Usern für Services und Programme die User \texttt{root}, \texttt{georg} und \texttt{webadmin}.

\item Der User \texttt{georg} gehört der Gruppe \texttt{admin} an. \texttt{webadmin} gehört den Gruppen \texttt{adm, dialout, cdrom, plugdev, lpadmin} \\  sowie \texttt{sambashare} an.
\end{itemize}

\end{document}

