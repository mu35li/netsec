\documentclass[10pt,a4paper]{article}
\usepackage[utf8]{inputenc}
\usepackage[german]{babel}
\usepackage[T1]{fontenc}
\usepackage{amsmath}
\usepackage{amsfonts}
\usepackage{amssymb}
\usepackage{minted}
\usepackage[left=2cm, right=2cm]{geometry}
\usepackage{url}


\setcounter{section}{1}
\author{Julian Deinert, Tronje Krabbe}
\title{Labreport 04}
\begin{document}
\maketitle
\tableofcontents
\newpage


\section*{1. Vertrautmachen mit der Umgebung}
\addcontentsline{toc}{section}{1. Vertrautmachen mit der Umgebung}
\setcounter{subsection}{1}

\subsection{}
Die SurfingVM hat die IP-Adresse \texttt{192.168.254.44}, wie man mit dem \textit{ip a} Befehl herausfinden kann.
Ihr Standard-Gateway ist \texttt{192.168.254.1}, gefunden mit \textit{ip r}.
Ihr Nameserver steht in \texttt{/etc/resolv.conf} und ist \texttt{10.1.1.1}.

\subsection{}
Die RouterVM benutzt auf dem Interface \texttt{eth0} die IP-Adresse \texttt{172.16.137.222},
und auf dem Interface \texttt{eth1} die Adresse \texttt{192.168.254.1}.
Die Adresse des von VMware zur Verfügung gestellten Gateways ist \texttt{172.16.137.2}.
All diese Informationen können wieder mit \textit{ip} herausgefunden werden.


\setcounter{section}{2}
\setcounter{subsection}{0}
\section*{2. Sniffing mit tcpdump}
\addcontentsline{toc}{section}{2. Sniffing mit tcpdump}
\setcounter{subsection}{1}

\subsection{}


\setcounter{section}{3}
\setcounter{subsection}{0}
\section*{Aufgabe 3}

\section*{Aufgabe 4}
\setcounter{subsection}{0}

\section*{Aufgabe 5}
\setcounter{subsection}{0}

\section*{Aufgabe 6}

\section*{Aufgabe 7}

\newpage
\section*{Appendix}
\end{document}
