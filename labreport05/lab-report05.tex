\documentclass[10pt,a4paper]{article}
\usepackage[utf8]{inputenc}
\usepackage[german]{babel}
\usepackage[T1]{fontenc}
\usepackage{amsmath}
\usepackage{amsfonts}
\usepackage{amssymb}
\usepackage{minted}
\usepackage[left=2cm, right=2cm]{geometry}
\usepackage{url}


\setcounter{section}{1}
\author{Julian Deinert, Tronje Krabbe}
\title{Labreport 04}
\begin{document}
\maketitle
\tableofcontents
\newpage


\section*{1. Netzwerkeinstellungen }
\addcontentsline{toc}{section}{1. Netzwerkeinstellungen}
\setcounter{subsection}{1}

\subsection{Ermitteln der Netzwerkkonfiguration}
\begin{itemize}
\item Die ClientVM hat die IP-Adresse \texttt{192.168.254.44} und das Standardgateway \texttt{192.168.254.2} außerdem verwendet sie den DNS-Server \texttt{10.1.1.1}.

\item Die RouterVM besitzt für das Interface \textit{eth0} die IP-Adresse \texttt{172.16.137.222} und für das Interface \textit{eth1} die IP-Adresse \texttt{192.168.254.2}.

\item Die ServerVM hat die IP-Adresse \texttt{172.16.137.144}.
\end{itemize}

\setcounter{section}{2}
\section*{2. Absichern eines Einzelplatzrechners mit iptables (ClientVM)}
\addcontentsline{toc}{section}{2. Absichern eines Einzelplatzrechners mit iptables (ClientVM)}
\setcounter{subsection}{0}
\subsection{Löschen aller Firewallregeln}

Wir richten die default policy wieder ein und flushen alle Chains  in der \textit{filter-}, \textit{nat-} und \textit{mangle-table}.\begin{minted}{php}
# iptables -P INPUT ACCEPT
# iptables -P FORWARD ACCEPT
# iptables -P OUTPUT ACCEPT

# iptables -t nat -F
# iptables -t mangle -F
# iptables -F
# iptables -X
\end{minted}
Danach installieren wir mit \texttt{apt-get} das Paket \texttt{openssh-server}.

\subsection{Entwerfen eines Konzepts}
Wir wollen Traffic durch Port \texttt{80} und \texttt{443} generell erlauben und Traffic durch Port \texttt{22} nur aus dem lokalen Netzwerk zulassen. Hierzu setzen wir die folgenden IP-Table Einträge:

\begin{minted}{php}

\end{minted}
\newpage
\section*{Appendix}
\end{document}
